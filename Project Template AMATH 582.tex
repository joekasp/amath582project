\documentclass[11pt]{article}
\usepackage{geometry}                % See geometry.pdf to learn the layout options. There are lots.
\geometry{letterpaper}                   % ... or a4paper or a5paper or ... 
%\geometry{landscape}                % Activate for for rotated page geometry
%\usepackage[parfill]{parskip}    % Activate to begin paragraphs with an empty line rather than an indent
\usepackage{graphicx}
\usepackage{amssymb}
\usepackage{epstopdf}
\usepackage{amsmath}
\usepackage{graphicx}
\graphicspath{{images/}}
\usepackage{float}
\usepackage{wrapfig}
\usepackage[titletoc, title]{appendix}
%\DeclareMathOperator{\sinc}{sinc}

\DeclareGraphicsRule{.tif}{png}{.png}{`convert #1 `dirname #1`/`basename #1 .tif`.png}

\title{Application of Compressive Sensing to Simulated Chemical Spectra}
\author{Camille Houferak, Joe Kasper, Joseph J. Radler, Shichao Sun}
\date{\today}                                           % Activate to display a given date or no date

\begin{document}
\maketitle
\begin{abstract}

Abstract

\end{abstract}

% SECTION I -- INTRODUCTION AND OVERVIEW

\section*{I.	Introduction and Overview}  %J will write this

%\begin{wrapfigure}{r}{0.38\textwidth}
%Sets wrapped figure to the right of the page
%\includegraphics[width=2.5in]{derek1}
%\caption{Corrupted Image of Derek's Trademark Look}
%\end{wrapfigure}

\par Computational chemistry seeks to construct fine-grained models of chemical systems in order to model and predict experimentally observable quantities in support of experimental research.  A major implementation of computational modeling is the construction of idealized chemical spectra, which are essentially the 'fingerprints' of a chemical species in a system of interest.  Traditionally, spectra have been generated experimentally, but recently have been generated from models of molecular electronic structure which comprises a major field of study in the computational chemistry community.  However, generating predictive spectra from model and simulation can be a difficult undertaking, as the number of degrees of freedom for a so-called \emph{full configuration} solution increases exponentially (or factorially!) for each additional electron found in the system. Full configuration calculations are ideal for capturing \emph{all} of the physics of a system, but come at great computational expense as every interaction between each particle must be accounted for in the Hamiltonian and the second-order differential \emph{Schr\"odinger Equation} solved numerically.  If atomic nuclear dynamics are also modeled, this increases the complexity even further as each solution for each time step.  Consequently, approximations and calculation schemes that minimize the complexity and computational cost of a model are highly valued in the field. Accordingly, techniques for reducing or removing the necessity of small time-steps and long timescales are also of great utility, as dynamical simulations examining nuclear motion often require picosecond timescales to capture the physics of interest whereas models examining only electronic dynamics require only femtosecond timescales.\\ 

\par Things are not so bleak, however, as there is an abundance of mathematical techniques for reducing the scale of such problems. This study focuses on one such technique called \emph{Compressed Sensing} (CS) that is used in data analysis to reduce the necessary sampling rate. CS does this by assuming a high degree of sparsity in a signal and makes use of the $L^{1}$ norm minimization technique to locate the vector of signal transform coefficients that is the most sparse.  

 As CS is frequently applied to time-frequency analysis problems, we apply it to modeled chemical spectra which heavily feature Fourier analysis for signal resolution. 
 
 %(This might not be relevant now) In the case of Infrared (IR) spectroscopy, the Fourier transform is used to present a frequency spectrum characteristic to a specific chemical species.  IR makes use of the vibrational frequencies of molecular bonds, which can be thought of as springs upon which the atoms oscillate in semiclassical picture, and the vibrational modes for different bond types are excited by specific IR frequencies of radiation. 


% SECTION II -- THEORETICAL BACKGROUND

\section*{II.	Theoretical Background}	%J and Camille	
\subsection*{A.	An Introduction to Chemical Spectra}

\subsubsection*{Optical Absorption Spectroscopy}	%Camille	

\subsection*{B.	Compressive Sensing}	%J
\subsubsection*{Theory}

\subsubsection*{Application to Chemical Spectroscopy}



% SECTION III -- ALGORITHM IMPLEMENTATION AND DEVELOPMENT
% J will take notes from Joe and Shichao to write this section

\section*{III.	Algorithm Implementation and Development}



% SECTION 4 -- COMPUTATIONAL RESULTS	
%Here we will examine the results of the study for each of the discrete transforms examined listed explicitly in subsubsections.  
%Included in each subsubsection will be the results of each of the sampling rates compared. 
\section*{IV.	Computational Results} %J and Camille using data and plots from Joe and Shichao

%IDEAL CASES

\subsection*{A.	Ideal Data Cases}
In order to test the validity of our CS algorithm for data reconstruction, we first applied it to idealized data sets represented by known linear combinations of sinusoidal functions. Having knowledge of the exact "signal" \emph{a priori} allows us to determine the sampling rates and discrete transforms that result in the best approximations under ideal conditions. 

%\begin{figure}[H]
%\includegraphics[scale = 0.8]{BWDerekVarSigma}
%\caption{Four different values of $\sigma$ applied to linear Gaussian filter to denoise \emph{Figure 3}}
%\end{figure}
%\pagebreak

% SIMULATED CASES

\subsection*{B.	Simulated Optical Absorption Spectra Cases}


%\begin{figure}[H]
%\includegraphics[scale = 0.8]{BWDerekVarSigma}
%\caption{Four different values of $\sigma$ applied to linear Gaussian filter to denoise \emph{Figure 3}}
%\end{figure}
%\pagebreak



%\begin{figure}[H]
%\includegraphics[scale = 0.8]{BWDerekVarSigma}
%\caption{Four different values of $\sigma$ applied to linear Gaussian filter to denoise \emph{Figure 3}}
%\end{figure}
%\pagebreak



%SECTION V. -- SUMMARY AND CONCLUSIONS
%We need to be sure to include our comparison of transforms and sampling rates with some sort of
%visualization in this section.  

\section*{V.	Summary and Conclusions} %J with input from Joe and Shichao 


\pagebreak
%REFERENCES----------------------------------------------------------------------------------------------
\section*{References}
[1]

[2]


\pagebreak
%APPENDICES-----------------------------------------------------------------------------------------------

\begin{appendices}

\section{MATLAB Functions and Implementations}

\subsection*{function subsection}

% MATLAB CODE SECTION ------------------------------------------------------------------------------
\section{MATLAB Code}
\begin{verbatim}
%Insert Matlab Code Here.  Whitespace is retained within {Verbatim}

\end{verbatim}

\section{Custom MATLAB Functions} 	%Use subsections for each one

\subsection*{Custom MATLAB Function 1}


\end{appendices}

\end{document}  